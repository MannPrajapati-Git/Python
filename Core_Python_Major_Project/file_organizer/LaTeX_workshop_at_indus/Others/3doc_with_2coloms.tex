\documentclass[10pt,twocolumn]{book}

\begin{document}

\title{Understanding LaTeX}
\author{A. B. Author}
\date{}

\maketitle

\chapter{Basics}
This book is typeset in two columns.
\section*{History of \LaTeX{}}

The origin of \LaTeX{} can be traced back to the development of \TeX{} by Donald Knuth in 1977. Frustrated by the poor quality of typesetting in his books, Knuth designed \TeX{} as a way to ensure high-quality output for technical documents. \TeX{} provides precise control over formatting but has a steep learning curve, especially for users unfamiliar with programming.

To make \TeX{} more user-friendly, Leslie Lamport developed \LaTeX{} in the early 1980s. \LaTeX{} introduced a layer of abstraction over \TeX{} by offering predefined commands and document structures. This enabled authors to focus on content rather than formatting details. Over time, \LaTeX{} gained popularity, and many publishers, academic institutions, and scientific communities adopted it as a standard for document preparation.

In the decades since its creation, \LaTeX{} has evolved with numerous extensions and packages that add powerful functionality, from graphics inclusion and bibliography management to slide presentations and multilingual support.

\section*{Basics of Using \LaTeX{}}

A basic \LaTeX{} document starts with the \verb|\documentclass| command, which specifies the type of document (such as article, report, or book). This is followed by a preamble section, where users can include packages and set document parameters. The content of the document is enclosed between \verb|\begin{document}| and \verb|\end{document}|.

\LaTeX{} uses markup tags similar to HTML. For example, sections are defined using commands like \verb|\section| and \verb|\subsection|. Text formatting commands include \verb|\textbf| for bold and \verb|\textit| for italics. Mathematical expressions are enclosed in dollar signs (\$...\$) for inline equations, or within \verb|\[ ... \]| for displayed equations.

One of the strongest features of \LaTeX{} is its ability to manage references and bibliographies automatically using packages like \texttt{biblatex} or \texttt{natbib}. Figures and tables can also be added using the \texttt{graphicx} and \texttt{tabular} environments.

\section*{Conclusion}

\LaTeX{} is an essential tool for researchers, scientists, and students who require precise control over document formatting. Its history is rooted in the need for quality typesetting, and its popularity continues to grow due to its flexibility, stability, and vast ecosystem of packages. Whether you're writing a thesis, a research article, or a book, \LaTeX{} offers the tools to create professional and polished documents.
\end{document}
